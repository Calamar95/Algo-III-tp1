\subsection{Idea general del problema}
Se ha decidido conectar telegráficamente todas las estaciones de un sistema férreo que recorre el país en abanico con origen en la capital (el kilómetro 0). Se nos ofrece cierta cantidad de kilometros de cable para conectar la ciudades de cada ramal. Al ser escaso el presupuesto, se busca lograr conectar la mayor cantidad de ciudades con los metros asignados, sin hacer cortes en el cable. \\
$~~~~~$Se nos propone resolver cuántas ciudades se pueden conectar para cada ramal, con una complejidad de O(n), siendo n la cantidad de estaciones en cada ramal.\\
$~~~~~$Para ello se nos brinda un archivo de entrada, el cual tiene para cada ramal dos líneas: la primera contiene un entero con los kilómetros de cable dedicados al ramal y la segunda los kilometrajes de las estaciones en el ramal sin considerar el 0. Luego de ejecutar nuestro algoritmo, la salida del mismo debe contener, para cada ramal de la entrada, una línea con la cantidad de ciudades conectables encontradas.\\

Un ejemplo de archivo de entrada puede ser, (extracto del archivo Tp1Ej1.in):\\
6 \\
6 8 12 15 \\
35 \\
8 14 20 40 45 54 60 67 74 89 99 \\
100 \\
35 87 141 163 183 252 288 314 356 387 \\
90 \\
6 8 16 19 28 32 37 45 52 60 69 78 82 \\

El mismo indica, en su primer línea que para el ramal 1 tenemos 6km de cable, y en su segunda línea que dicho ramal contiene (además de la capital, implícita, en el kilómetro 0) una estación en el kilómetro 6, otra en el 8, otra en el 12 y la última en el kilómetro 15. Luego para el ramal 2, tenemos 35 kilómetros de cable, y estaciones en los kilómetros: 8 14 20 40 45 54 60 67 74 89 y 99. Así sucesivamente para el resto de los ramales.\\
$~~~~~$El archivo de salida luego de ejecutarse nuestro algoritmo deberá ser de la siguiente pinta, (extracto del archivo Tp1Ej1.out):\\
3 \\
6 \\
4 \\
14 \\

Este último archivo indica la cantidad de ciudades que se pueden conectar para cada ramal. En el caso del ramal 1, para el cual se tienen 6km de cable disponibles, y contiene ciudades en los kilómetros: 0 6 8 12 15 vemos que la solución debería ser que se pueden conectar como máximo 3 ciudades, a continuación explicaremos cómo se deduce esto.\\

Si conectamos la capital con la ciudad del kilómetro 6, al tener sólo 6km de cable, nuestra solución sería que pudimos conectar sólo 2 ciudades. Pero como debemos maximizar esta cantidad, podemos ver que si en vez de conectar a la capital con la primer estación del ramal, conectamos la ciudad del kilómetro 6, con su siguiente y con la del kilómetro 12, entonces como entre el kilómetro 6 y el 8 hay una diferencia de 2kms y entre el 8 y el 12 una diferencia de 4kms, vemos que la máxima cantidad de estaciones conectadas con 6km de cable para el ramal 1, es 3. La misma lógica se la aplica para los ramales restantes.\\

\subsection{Explicación y pseudocódigo}

El algoritmo empieza ubicando dos índices llamados ``start" $ $y ``actual" $ $ en la primera ciudad e inicializando ``restemp" $ $ con el valor 0 (a lo largo del algoritmo esta variable contendrá la solución parcial del ejercicio). Luego, si el cable es suficientemente largo, avanza ``actual"$ $ hasta la segunda ciudad, y le resta al cable esa distancia (entre la primer ciudad y la segunda). Cada vez que una nueva ciudad es conectada, la variable ``conectadas" $ $ aumenta en uno, al igual que el índice ``actual"$ $. De esta manera, continúa (restando la distancia entre la tercera y la segunda, luego entre la cuarta y la tercera, etc.) hasta que ya no se puedan conectar más ciudades, es decir, hasta que el cable se acaba, o en caso de llegar a la última ciudad (cuando ``actual"$ $ apunta a la última ciudad).  Si llegamos a la última ciudad, el algoritmo verifica si la cantidad de ciudades del ramal recién calculado (es decir, ``conectadas" $ $) es mayor que ``restemp" $ $; lo reemplaza en caso de que si, y retorna ``restemp" $ $, que es el resultado final. En caso de que el cable se acabe, calcula la distancia entre la ciudad apuntada por ``start" $ $ y la siguiente ciudad; suma esa distancia al resto del cable que quedo disponible y avanza el índice ``start" $ $ a la siguiente ciudad del ramal, para ver si se puede construir otra conexión con una mayor cantidad de ciudades. Además, verifica si ``conectadas" $ $ es mayor que ``restemp" $ $, lo reemplaza en caso positivo, y luego le resta uno a dicha variable (siempre que esta sea mayor a 0), pues esto simboliza que una ciudad fue desconetada. Con esta nueva cantidad de cable disponible, los pasos recien descriptos se repiten, y se verifica nuevamente si se puede incluir la ciudad siguiente (la siguiente a ``actual"$ $) al ramal. El algoritmo finaliza una vez que se llegue a la última ciudad.

A continuación mostramos un pseudocódigo de la implementación explicada anteriormente. \\

int conectar(vector$<$int$>$ v , int cable) \{ \\
$~~~~~~~~~~~~$int resTemp $\leftarrow$ 0 \Ode{1}\\
$~~~~~~~~~~~~$int start $\leftarrow$ 0 \Ode{1}\\
$~~~~~~~~~~~~$int actual $\leftarrow$ 0 \Ode{1}\\
$~~~~~~~~~~~~$int aux $\leftarrow$ 0 \Ode{1}\\
$~~~~~~~~~~~~$\textbf{mientras} (actual no apunte al último elemento del vector) \{   \Ode{n} \\
$~~~~~~~~~~~~~~~$\textbf{mientras}(cable $\geq$ 0 y actual no apunte al ultimo elemento del vector) \{  \Ode{cable} \\
$~~~~~~~~~~~~~~~~~~~~$Guardamos en aux el cable, por si este pasa a ser negativo \Ode{1}\\
$~~~~~~~~~~~~~~~~~~~~$Le restamos al cable la distancia entre lo apuntado por actual y su siguiente; \Ode{1}\\
$~~~~~~~~~~~~~~~~~~~~$\textbf{si} (longitud del cable $\geq$ 0) \{  \Ode{1}\\
$~~~~~~~~~~~~~~~~~~~~~~~~~~$Incrementamos conectadas en 1 y avanzamos el puntero actual \Ode{1}
$~~~~~~~~~~~~~~~~~~~~$\} \\
$~~~~~~~~~~~~~~~~~~~~$Si conectadas es igual a uno se lo cambia por dos;  \Ode{1}\\
$~~~~~~~~~~~~~~~~$\}\\
$~~~~~~~~~~~~~~~$Si el cable en negativo, se actualiza su valor por el de aux\\
$~~~~~~~~~~~~~~~~~~$Si conectadas es mayor que restemp se actualiza el valor de restemp por el de conectadas\\
$~~~~~~~~~~~~~~~$\textbf{si}(resTemp == 0) \{  \Ode{1} \\
$~~~~~~~~~~~~~~~~~~~~$Conectadas  $\leftarrow$ 0; \Ode{1}\\
$~~~~~~~~~~~~~~~~~~~~$Avanzamos los dos índices, start y actual;  \Ode{1}\\
$~~~~~~~~~~~~~~~$\} \textbf{si no} \{ \\
$~~~~~~~~~~~~~~~~~~~~$Decrementamos conectadas en 1, ya que desconectamos la primer ciudad \\
$~~~~~~~~~~~~~~~~~~~~$Sumamos al cable la distancia entre lo apuntado por start y su siguiente;\de{1}\\
$~~~~~~~~~~~~~~~~~~~~$Incrementamos start en 1;  \Ode{1}\\
$~~~~~~~~~~~~~~~$\} \\
$~~~~~~~~~~~~$\} \\
$~~~~~~~~~~~~$Devolvemos restemp, en caso de que sea uno, se cambia su valor por dos; \Ode{1} \\
\}\\

\subsection{Deducción de la cota de complejidad temporal y correctitud}

A partir del siguiente análisis de nuestro algoritmo, deducimos que la complejidad del mismo para resolver el problema dado es de O(n) en el peor de los casos, donde n es la cantidad de estaciones en cada ramal. \\

El algoritmo que construimos consta de dos ciclos: el primero se ejecuta a lo sumo n veces, y eso ocurre cuando tenemos un cable demasiado corto que no alcanza para unir ni siquiera dos ciudades. Lo que ocurre en este caso, es que los dos índices (``start"$ $ y ``actual"$ $) se van incrementando en uno, recorriendo así todo el vector cada índice, y el segundo ciclo solo iteraría una vez pues el cable pasaría a ser negativo instantaneamente. También podría ocurrir que el primer while se ejecutarse solo una vez (si se da un cable muy largo); en este caso el segundo while es el que iteraría n veces, llevando el puntero ``actual"$ $ hasta el final, pero sin mover ``start"$ $ por lo que seria considerado como el mejor caso, aunque este también tiene una complejidad O(n).\\

En otros casos, el segundo while del algoritmo se ejecuta mientras que la longitud del cable sea positiva y el índice ``actual"$ $ no apunte a la última ciudad; este ciclo va sumando ciudades al ramal y guarda cuántas ciudades se conectaron. Mientras que va conectando ciudades, el algoritmo hace avanzar el segundo índice (``actual"$ $) y, cuando el cable pasa a ser negativo, se incrementa el primer índice ``start"$ $, se suma esa diferencia al cable, se verifica si hay que actualizar el resultado, y vuelve a comenzar el segundo ciclo. Este proceso continúa hasta que ``actual"$ $ alcance a la última ciudad. Notar que en cada iteración, ``actual"$ $ no siempre avanza, y cuando no lo hace, ``start"$ $ sí. Si ocurre que ``start"$ $ y ``actual"$ $ se superponen, entonces en la iteración siguiente, si ``start"$ $ vuelve a avanzar, lo empuja(es decir, ``actual"$ $ también avanza). Aquí se puede obervar el peor caso, que ocurre cuando ``actual"$ $ llega hasta cierto punto arbitrario, y luego ``start"$ $ lo alcanza. Si esto ocurre constantemenete, se recorre linealmente el vector dos veces, y aunque sigue siendo O(n), es el caso que más iteraciones produce y por eso es considerado el peor caso.\\

El algoritmo termina una vez que el segundo puntero llega al final. Como ninguno de los dos índices retrocede nunca, y si ``start"$ $ alcanza a ``actual"$ $, en la siguiente iteración ``actual"$ $ avanza,  se puede deducir que el algortimo es lineal y por lo tanto tiene una complejidad O(n). Por esta razón decimos que nuestro algoritmo es correcto, ya que siempre llega al final, siempre devuelve el resultado esperado y respeta la cota del enunciado. \\

Las funciones que usamos: $>$, $<$, $\geq$, $+$, $-$, $==$, tienen complejidad constante al igual que las asignaciones. También sabemos por la documentación de C++, que el $operator[]$ del vector es de complejidad O(1) al igual que la función del vector $.back()$.\\

\subsection{Casos de test, experimentación, y gráficos}

\subsubsection{Caso random}
Vamos a mostrar la implementación de un test generado sin ninguna intencionalidad, pero antes explicaremos como fue creado. \\

Implementamos una función que genera números random para el archivo que le vamos a pasar por entrada a nuestro algoritmo con los siguientes criterios:
\begin{itemize}
\item Elegimos en este ejemplo que el primer ramal iba a contar con una estación, el segundo con 101, el tercero con 201 y asi sumando de a 100.
\item Para el valor de la longitud de cable disponible de cada ramal, decidimos que sea un número random entre 1 y la cantidad de estaciones de cada ramal.
\item Para los kilometrajes de las estaciones tuvimos en cuenta, que estos debían estar ordenados de menor a mayor, sin contener el kilómetro 0. 

El archivo que contiene dicha función se puede encontrar en la carpeta Ejercicio 1, y se llama: generadorArchivosInRandom.py.

\end{itemize}

Una vez que generamos el archivo del input con dicho test, ejecutamos nuestro algoritmo obteniendo el archivo de salida con la cantidad de ciudades conectadas para cada ramal, e imprimimos por pantalla el tiempo promedio de 100 corridas de nuestro algoritmo, en milisegundos. Este es el tiempo que tarda en calcular la máxima cantidad de estaciones conectadas de cada ramal. Decidimos calcular el tiempo promedio porque al ejecutarlo un par de veces nos dimos cuenta que el tiempo varía aunque resuelva la misma instancia, entonces decidimos correrlo una cierta cantidad de iteraciones (en este caso 100), e ir acumulando los tiempos para luego dividir este acumulador por 100 y así obtener un valor promedio de los tiempos en los que se tarda en resolver el problema para cada ramal. \\

Con estos tiempos creamos el gráfico de la figura \ref{ej1-random} que mostramos a continuación, en el que hacemos una comparación con la gráfica de O(n) y observamos como nuestro algoritmo cumple dicha complejidad.

\begin{figure}[H]
\begin{center}

\minipage{0.8\textwidth}
  \includegraphics[width=\linewidth]{../graficos/ej1/CasoRandom.png}
  \caption{{\small Comparación con O(n). Dado un archivo de entrada con longitud de cable y kilometrajes de estaciones random.}} \label{ej1-random}
\endminipage

\end{center}
\end{figure}


\subsubsection{Peor caso}

El próximo test generado consiste en tener una ciudad al final muy lejos, lo que produciría que ``actual" $ $ llegue hasta el anteúltimo, luego ``start" $ $  lo alcance y esto produzca que se pase por todas las ciudades dos veces. Esto lo consideramos un Peor caso ya que para el resto de los casos no hace falta pasar dos veces por todas las ciudades.  Para probar esto, escribimos un script que nos asegura que la última ciudad este muy alejada. En cuanto a las estaciones del ramal, configuramos números random y el ultimo muy grande. Estos archivos se encuentran en la carpeta ``Ejercicio 1" $ $ con prefijo ``5toEjemplo" $ $, y``generadorArchivosInRandom.py" $ $es el script con los distintos tests, comentados. Veamos un ejemplo:\\

\begin{itemize}
\item La longitud del cable llega hasta la anteúltima ciudad, y la última ciudad esta muy lejos:\\
\textbf{in:}\\
90\\
3 6 8 12 15 20 24 49 58 70 90 1000 \\
\textbf{out:}\\
12\\

En este caso la longitud del cable es 90, por lo tanto está bien que nuestro algoritmo devuelva que conectó 12 ciudades.\\
\end{itemize}

\begin{figure}[H]
\begin{center}

\minipage{0.8\textwidth}
  \includegraphics[width=\linewidth]{../graficos/ej1/PeorCaso.png}
  \caption{{\small Comparación con O(n). Dado un archivo de entrada con un cable suficientemente largo como para llegar a la anteúltima ciudad pero con una última ciudad muy lejana, lo que genera que se avancen los dos índices.}} \label{ej1-peor}
\endminipage

\end{center}
\end{figure}

En la figura \ref{ej1-peor} vemos como nuestro algoritmo sigue respentando el límite de complejidad propuesto por la cátedra, a pesar de que éste sea un caso ``border" $ $. \\

\subsubsection{Mejor caso}

Otro test que decidimos hacer fue el que la longitud del cable es tan larga que permite conectar todas las ciudades de los ramales. Esto lo consideramos como el mejor caso, ya que al tener un cable suficientemente largo como para conectar todas las ciudades, el primer puntero ($"$start$"$) no es necesario que se mueva, solo avanza el segundo. Para ver esto , escribimos un script de manera tal que la cantidad de kilómetros de cable siempre supere la distancia entre la primera ciudad y la ciudad del último kilómetro del ramal. Al medir los tiempos promedios vemos que nuestro algoritmo sigue tardando menos que O(n), y lo podemos observar en la figura que sigue, pero antes veamos un ejemplo. \\

\begin{itemize}
\item El cable alcanza para conectar todas las ciudades:\\
\textbf{in:}\\
10000\\
1 4 67 78 90 95 120 150 270 380 456 900 1300 1809 5546 8403\\
\textbf{out:}\\
17\\
\end{itemize}
En este caso al tener un cable de longitud = 10000km y todas las estaciones estar a distancia menor que 10000km, el algoritmo devuelve 17 que son la cantidad de estaciones del ramal + la ciudad de kilómetro 0, pues la distancia entre ésta ciudad y la primera es de 1km. Veamos el gráfico \ref{ej1-mejor}. \\

\begin{figure}[H]
\begin{center}

\minipage{0.8\textwidth}
  \includegraphics[width=\linewidth]{../graficos/ej1/MejorCaso.png}
  \caption{{\small Comparación con O(n). Dado un archivo de entrada con un cable muy largo y kilometrajes de estaciones random.}} \label{ej1-mejor}
\endminipage

\end{center}
\end{figure}

\subsubsection{Mejor caso vs Caso random vs Peor caso vs O(n)}

A continuación ilustramos un gráfico (figura \ref{ej1-todos}) en donde comparamos el tiempo que tarda el algoritmo en el mejor caso, en el caso randon, y en el tiempo del peor caso, y con O(n). Para realizar el gráfico definimos la línea O(n) como n $*$ 0,009.

\begin{figure}[H]
\begin{center}

\minipage{0.8\textwidth}
  \includegraphics[width=\linewidth]{../graficos/ej1/TodosLosCasos.png}
  \caption{{\small Comparación con O(n) del mejor, peor y caso random anteriormente explicados. }} \label{ej1-todos}
\endminipage

\end{center}
\end{figure}


\subsubsection{Otros casos interesantes}
\begin{itemize}
\item Cuando el cable alcanza para conectar solo las primeras ciudades:\\
\textbf{in:}\\
9\\
1 2 3 4 5 40 55 68 79 99 130 139 200 259 2889\\
\textbf{out:}\\
6\\

\end{itemize}

Al tener un cable de longitud 9km, las primeras 5 estaciones estar a 1 kilómetro de distancia entre ellas, y las siguientes distancias superar los 9km, si contamos estas 5, y a la ciudad del kilómetro 0, no devuelve las 6 ciudades que conecta.\\

\begin{figure}[H]
\begin{center}

\minipage{0.8\textwidth}
  \includegraphics[width=\linewidth]{../graficos/ej1/PrimerasCiudades.png}
  \caption{{\small Comparación con O(n). Dado un archivo de entrada donde entre las primeras ciudades hay una distancia corta.}} \label{ej1-primerasciudades}
\endminipage

\end{center}
\end{figure}

\begin{itemize}
\item Cuando el cable no alcanza y no se conectan ciudades:\\
\textbf{in:}\\ 
1\\
4 8 12 16 20 24 28 32 \\
\textbf{out:}\\
0\\

La distancia entre todas las estaciones es de más de 1km, y el cable tiene un largo de 1km, por lo tanto, al faltarle siempre cable para poder conectar al menos dos ciudades, devuelve 0. Esto es correcto ya que no alcanza el cable y no ``conecta" $ $ ciudades. \\

A continuación mostramos el gráfico \ref{ej1-cablecorto} que resulta de medir el tiempo que tarda el algoritmo al correr un archivo de entrada que generamos con ciudades que están a una distancia mayor a un, con un cable de longitud 1 y compararlo con O(n), en particular para este ejemplo, lo comparamos con 0,009 $*$ n. Los archivos que usamos son: ``generadorArchivosInRandom.py" $ $ (ejemplo 6), ``2doEjemploTiempoMilisegundos.txt"  $ $, \\ ``2doEjemploCableCorto.in"  $ $, ``2doEjemploCableCorto.out"  $ $ y , ``ejemploTamCiudades.txt"  $ $.\\

\end{itemize}

\begin{figure}[H]
\begin{center}

\minipage{0.8\textwidth}
  \includegraphics[width=\linewidth]{../graficos/ej1/CableCorto.png}
  \caption{{\small Comparación con O(n). Dado un archivo de entrada donde se da un cable corto para conectar las ciudades.}} \label{ej1-cablecorto}
\endminipage

\end{center}
\end{figure}

